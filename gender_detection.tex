
\documentclass[runningheads]{llncs}
\usepackage{graphicx}


\begin{document}

	\title{Gender Detection on the Web}
	\titlerunning{Gender Detection on the Web}
	\author{Xiaoqi Ma}
	\institute{RWTH Aachen University \\
		\email{xiaoqi.ma@rwth-aachen.de}\\}
	\maketitle  

	\begin{abstract}
		The abstract should briefly summarize the contents of the paper in
		15--250 words.
		
		\keywords{First keyword  \and Second keyword \and Another keyword.}
	\end{abstract}
	\section{Introduction}
	\subsection{A Subsection Sample}

	Please note that the first paragraph of a section or subsection is \cite{karimi2016inferring}
	not indented. The first paragraph that follows a table, figure,
	equation etc. does not need an indent, either.
	
	Subsequent paragraphs, however, are indented.
	
	
	\begin{table}
		\caption{Table captions should be placed above the
			tables.}\label{tab1}
		\begin{tabular}{|l|l|l|}
			\hline
			Heading level &  Example & Font size and style\\
			\hline
			Title (centered) &  {\Large\bfseries Lecture Notes} & 14 point, bold\\
			1st-level heading &  {\large\bfseries 1 Introduction} & 12 point, bold\\
			2nd-level heading & {\bfseries 2.1 Printing Area} & 10 point, bold\\
			3rd-level heading & {\bfseries Run-in Heading in Bold.} Text follows & 10 point, bold\\
			4th-level heading & {\itshape Lowest Level Heading.} Text follows & 10 point, italic\\
			\hline
		\end{tabular}
	\end{table}
	
	\begin{table}
		\begin{tabular}{|l|l|l|}
			\hline
			 Features &  Example & \#female/\#male\\
			\hline
			Emotions &  :), :D & 3.5\\
			Elipses &  ....  & 1.5\\
			Character repetition & niceee, no waaay & 1.4\\
			Repeated exclamation & !!!!  &  2.0 \\
			Puzzled punctuation & !?!!??! & 1.8\\
			OMG & Oh My God & 4.0\\
			\hline
		\end{tabular}
	\end{table}
	
	\noindent Displayed equations are centered and set on a separate
	line.
	\begin{equation}
	x + y = z
	\end{equation}
	Please try to avoid rasterized images for line-art diagrams and
	schemas. Whenever possible, use vector graphics instead 

	\section{Related Work}
	
	\section{Methods}
	
	\section{Results \& Discussion}
	
	\section{Implications}
	
	
	% prevents nocite errors
	\nocite{*}
	\bibliography{literature}
	\bibliographystyle{IEEEtran}
	

\end{document}